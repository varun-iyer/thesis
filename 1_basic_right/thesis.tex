\documentclass[12pt]{article}
\usepackage{mathpazo}
\usepackage[backend=bibtex]{biblatex-chicago}
\bibliography{thesis}

\begin{document}

\section{The Basic Right}
	This work seeks to examine the moral principles which human and non-human
	coexistence on the planet.  But before venturing into more uncertain
	territory, I will begin by sketching out a starting point:

	\begin{quote}
		\emph{The Basic Right:} all animals have the right not to \emph{used}
	as property, resources, or as mere means to an end.  \end{quote}

	In this section, I’ll provide some arguments reasons to believe that the
	basic right exists and touch on how to determine which things are owed this
	right (which things are animals).  This section won’t contain an exhaustive
	argument for this right --- several other works have already painstakingly
	laid out the arguments to support this basic right for animals.  I will
	conclude by describing what implications this fundamental right has, and
	which implications it does not have.

	\subsection{Grounding the Right}
		
	For the purposes of this work, I take the view of fundamental rights
	advanced by Christine Korsgaard in \emph{Fellow Creatures}.\autocite{kors}

	Korsgaard’s argument is established on the foundation of Kant’s moral
	philosophy.  Kant seeks to determine the presuppositions of valuing; the
	premises that are implicitly accepted any time we confer value upon
	something.  There is no metaphysical “absolute” reference frame from which
	“true” value can be determined.  Rather, things are valuable insofar as
	they are valuable to a being.

	Kant argues that rational beings can only pursue ends that are absolutely
	good, good from all perspectives.\autocite[8.4.1]{kors} So, an end that one
	rational being pursues is absolutely good and worthy of pursuit by any
	other.  Humans have the capability to act as rational beings.  When humans
	rationally act on ends, they confer absoulute value on those ends, marking
	them as good absolutely.  “By pursuing what is good for you as good
	absolutely, you show that you regard yourself as an end in itself, or
	perhaps to put it a better way, you make a claim to that
	standing.”\autocite[8.4.4]{kors}

	This seems to leave animals --- most of whom are not rational beings ---
	out of the moral picture.  If animals are not rational, then they cannot
	make a claim to standing as an end in themselves.

	Korsgaard argues that Kant uses the phrase ‘end in itself’ to refer to two
	slightly different concepts.  Kant somtimes refer to an ‘end in itself’ as
	a being who has the ability to legislate for itself and all rational
	beings.  At other times, Kant refers to an ‘end in itself’ as a being whose
	ends are recognized as absolutely good and protected by universal
	legislation.  Korsgaard argues that the two do not always need to be one
	and the same.

	When we act as rational beings, we do not assert that only rational beings
	have value.  After all, a rational being without any substance, form, or
	other natures would not have any ends within itself to seek or any desires
	to pursue.  Instead, our rational nature confers value upon the ends we
	seek as animals, creatures who have representations of the world and seek
	good within it.  We share these ends and this nature with other non-human
	animals, and so when we value ourselves as creatures that have a final
	good, we also confer value on other creatures with final goods.  In her own
	words,

	\begin{quote} As rational beings, we need to justify our actions, to think
		there are reasons for them. That requires us to suppose that some ends
		are worth pursuing, are absolutely good. Without metaphysical insight
		into a realm of intrinsic values, all we have to go on is that some
		things are certainly good-for or bad-for us. That then is the starting
		point from which we build up our system of values—we take those things
		to be good or bad absolutely—and in doing that we are taking ourselves
		to be ends in ourselves.  But we are not the only beings for whom
		things can be good or bad; the other animals are no different from us
		in that respect. So we are committed to regarding all animals as ends
		in themselves.\autocite[8.5.5]{kors} \end{quote}

		\subsection{Coming Apart from Korsgaard}

		Thus far, it has been established that animals are ends in themselves.
		But this alone does not show the principle laid out in The Basic Right.
		We must ask what it means to be treated as an end, and how treatment as
		an end in the rational, legislative sense might differ from treatment
		as an end in the human sense.

		Korsgaard comes to the conclusion that animals have a good, but because
		they lack rationality, they do not get an ‘equal vote’ in our
		interactions with them.\autocite[12.2.1]{kors} Despite the fact that
		they cannot consent to our interactions with them, because our own
		moral legislation declares their ends to be worthy, we have a duty to
		treat animals in a manner that is consistent with their
		ends.\autocite[12.2.1]{kors} Korsgaard distinguishes this treatment
		from treatment of animals “in ways to which they would consent if they
		could,” but she is unclear if the two generate different
		outcomes.\autocite[12.2.1]{kors}

		However I think that Korsgaard’s conclusions entail the principle of
		the Basic Right, and that respect of the Basic Right is a subset of
		Korsgaard’s duties to treat animals in a manner consistent with their
		ends.

		\subsubsection{Consistent with Consent and Consistent with Good}

			A creature is an end in itself by virtue of the fact that it has a
			final good which it seeks through action.  What is good-for the
			creature are things which the creature seeks by acting.  Our
			knowledge of the good of creatures is only gained by observing the
			creature’s actions; what the creature seeks and avoids.  When we
			assert that something is good for a creature, we assert that it is
			a thing that the creature would seek out or act towards.  When we
			interact in ways that are consistent with a creature’s good, we
			interact in ways that are not contrary to things that are good for
			the creature --- things that we think the creature would seek out
			as a final good.

			This doesn’t mean, of course, that we must always act in ways that
			creatures would --- it is permissible, after all, to take a cat to
			the vet.  But taking a cat to the vet against their wishes is
			different from attempting to make a cat appreciate classical music,
			or choosing to kill the cat for use as coat-liner.  In the former
			case, we take something we know the cat would act towards (good
			health), and intervene to use our greater knowledge of the
			situation (the vet is there to help) to help secure the cat’s own
			final good.  The euthanization case that Korsgaard gives may be
			something similar; a decision we can permissibly make for a
			creature because we have a greater knowledge of what death
			is.\autocite[12.2.1]{kors} In the latter cases, we seek to impose
			some other good on the cat; goods that a cat would not act towards
			themselves.

			Because our knowledge of a creature’s good is defined by their
			actions and pursuits, I don’t think there’s very much daylight
			between Korsgaard’s two moral guidelines of acting in ways that an
			animal could consent and acting in ways that are consistent with an
			animal’s good.  The one place where they may come apart is in
			treatment that would undermine a being’s autonomy while still
			satisfying their good.  For a rational being, this class of actions
			does not exist, because a rational being has no good other than
			their autonomous choices.  Non-rational ends such as creatures mark
			their final goods as absolute, but do not legislate for others
			through their actions.

			\marginpar{\footnotesize \emph{Note 1:} change this slightly to say that creature’s good and ways they would consent are the same; don’t make this distinction}
			Deception, for example, is wrong because it undermines a rational
			being’s autonomy, and it is not something a rational being could
			consent to. Non-rational beings cannot consent to treatment, but
			define their good through their actions over time. It would be
			permissible to deceive a cat (by offering a treat, or lying about a
			travel destination) to get them to go to the vet, as long as the
			ultimate aim of the action was consistent with the good of the cat.

		\subsubsection{Consistent with Good and Basic Right}
			
			I believe that, similarly, there is a narrow window between
			treating an animal in a manner which is consistent with their good
			and not treating an animal as a means to an end.  I think that this
			gap is essentially close-able, depending on one’s view of the moral

	\subsection{Epistemology of Subjectivity}
	\subsection{Implications}
	\subsection{Uncharted Territory}

\section{Animals and Property}
\end{document}
