\documentclass[12pt]{book}
\usepackage{mathpazo}
\usepackage[backend=bibtex]{biblatex-chicago}
% \bibliography{../thesis}

\begin{document}

\chapter{Animals and Property}

    In her work \emph{Moral Status}, Mary Anne Warren argues that animals have
    a lower moral status than humans.\footnote{Warren is a little dated...
    should I be bringing her up here?} Warren's argument relies on the intuition
    that we are willing to kill rats that invade our homes where we would not
    be willing to kill human children if they were in the same position.
    
    Warren's argument has already been rejected\footnote{that one 2007 paper},
    but her hypothetical raises unanswered questions. Human and animal interests
    come into constant conflict over the Earth's land and resources. It seems as
    if any hard-line animal rights theory would be forced into an uncomfortable
    conclusion. To minimize killing animals over land or resource
    conflicts, humans would be forced to return to gatherer lifestyles.
    
    In the previous section, I concluded that it is impermissible to use
    animals as mere resources for human ends. In this section, I will consider
    the duties we have to share the resources of the Earth with other animals.

	Property, or something like it, is a pre-theoretical and pre-human notion.
	A beehive as the property of the bees who live in it.  Marked territory
	seems like a kind of property that a wolf seeks to claim.  And I am
	certainly unlikely to intrude on a bear in their hibernation den.
	\texttt{TODO citations/explanations from Wildlife as Property Owners}.  If
	humans claim special protections and a right to defend their own property,
	then I don’t think that it’s unreasonable to think that animals might have
	similar, special rights to their own as well.
    
    I will conclude that humans and animals both have subject-relative 
    prerogatives to defend their rightfully acquired property. This prerogative
    permits the use of lethal defensive force as a final resort to defend
    a person's closely held property against nonhuman and human invaders alike.
    However, there are circumstances that might render humans liable to 
	defensive force or intrusions by other beings.
	
    Furthermore, I will argue that humans do not have a duty to re-wild
    long-held human spaces for the sake of other animals, but that all people
    have strong duties to reduce or eliminate the harm caused through
    continuing human expansion.
    
\section{Defense of Property}

	Defense of property sometimes justifies actions that are otherwise
	impermissible. If a person is walking across the street in front of us, we
	may not demand that they leave. We certainly should not harm them or treat
	them as liable to harms.  We can certainly demand that a person walking
	around our home leave. If they ignore the demand, they may be liable to
	harm.

	These statements are relatively uncontroversial when referring to rational
	persons who willingly and knowingly trespass on private, individually owned
	or occupied property. This becomes less clear when discussing nonhuman
	animals. Few of them could be described as rational and even fewer might
	understand the concept of property. Even if they could, they might fall
	outside the covenant or agreement that protects property rights.

	However, it seems like the concept of self-defense or defense of
	property can still be applied against animals.  If there are insects living
	in my skin, I feel no qualms about killing them, even though they may not
	be aware they are invading my body.\footnote{I will assume for the moment
	that these insects should be accorded moral status though it is unclear.}
	I am also willing to believe that it is easier to justify capturing or even
	killing rats in one’s home rather than rats living in the woods.

	In this section, I’ll consider the nature of the right of property and the
	morality of defense of property. In the following section, I will expand
	on this baseline and apply the 

	\subsection{Claiming Property}

	Property are the things that a being controls in order to secure their
	ends.  To hold a resource as property is to have a rightful claim to use it
	for any purpose that does not interfere with the choice or similar right of
	another.  To claim an unclaimed resource as property, one need only
	empirically control it and signal an intent to continue controlling it.
	The concept of property is not dependent on the existence of an agreement to
	enforce or define it.\footnote{Perhaps I should make this section more
	explicitly Kantian? It’s based closely on foundations in Ripstein and
	descriptions in Quong}

	Consider a pared down example. A table is laid out with a collection of
	books on it, along with a sign saying “Free to Take.” I see a book I like,
	and I intend to collect it after class. If another person takes the book
	before I do, then they have not wronged me.

	On the other hand, imagine I actually did take the book. Another person
	sees it in my bag and chooses to take it from me. This person has wronged
	me because they used an object which I had marked as my property in a way
	that I did not authorize.

	This example also demonstrates that an interest-based theory of property
	alone is insufficient. In both cases, my interests are set back by equal
	amounts. However, I am only wronged in the latter case.

	This description is rather simple and far from complete. I will readily
	admit that there are certain restrictions in claiming property --- it would
	probably be wrong for a person to take all of the books, for example. Rights
	of property can be modified or clarified by political bodies. In addition,
	there is a distinction between property (specific pieces of real property,
	objects, and agreements) and wealth (the degree that the sum total of one’s
	property allows them to secure their ends). While there ought to be
	relatively stringent protections over individual pieces of property
	(something like eminent domain/takings requirements), enforceable
	obligations to transfer a certain amount of wealth need not be as strongly
	justified and compensated.

	\subsection{Property in a State of Nature}

	Even within a state of nature, prior to any unification of will or
	hypothetical agreements, intruding upon a person’s rightfully claimed
	personal property renders the intruder liable to defensive harm unless the
	property owner is duty-bound to accept the intrusion.  Establishing that
	property may be defended in a state of nature is important because we are
	(quite literally) in a state of nature with the other animals. 

	Consider an example given by Jonathon Quong.  A shipwrecked sailor washes
	up to the shore of a deserted island. Her leg is wounded. In order to save
	a limb, she can use a special piece of seaweed that has washed up on shore
	with her. She uses the seaweed to cover the wound. Soon after, another
	person washes up alongside. The second person also requires the seaweed to
	staunch a wound. The second person has an interest in the seaweed that is
	equal to the interest of the first person. It seems clear here that the
	second person could not take the seaweed from the first.\footnote{Quong}

	To extend the example, lets say that the sailor needs to set the seaweed
	aside temporarily. If she demonstrates a need and intent to continue using
	it, then she has demarcated it as her property. The second person ought not
	take the seaweed from the first even though it is no longer directly
	connected to her person. If the second person does so, then they render
	themselves liable to proportional defensive harm from the first.

	In a pre-political state, there is no assurance that a right to property
	will be respected.\footnote{Ripstein} In a way, a person is still subject
	to the choices and good graces of others. Remaining in this state would be
	wrong for all rational beings. As Japa Pallakathiyil notes, a lack of
	assurance does not necessarily imply that a person’s right loses all
	\emph{moral} force. Pallakathiyil goes on to argue that the state of nature
	renders a persons right inconclusive, rendering their use of defensive
	force impermissible. I find this argument lacking.\footnote{I could break
	out into a longer discussion of Pallakathiyil here but I don’t want to but
	it seems like I really should.} My intuition is that it remains permissible
	to use defensive force over clearly determined property even in the absence
	of a political body.

	\subsection{The Agent-Relative Prerogative}

	In the previous section, I stated that a person can become liable to
	proportional defensive harm when they intrude on another person’s property.
	This is only applicable when a person is rational (can respond to moral
	reasons) and aware that the object they are interacting with is another’s
	property.\footnote{For example, a rational elephant may not become liable
	to defensive harm because they may not be aware of the kinds of signals
	used by humans to claim property.}

	This position is insufficient to explain situations where the intruder is
	not rational or responsible for their intrusion --- the innocent attacker.
	This situation describes most intrusions by non-human animals, all of whom
	are nonrational or unfamiliar with human conventions and signs that claim
	property.

	I will adopt Jonathon Quong’s explanation of the morality of defense against
	an innocent attacker. Quong’s explanation relies on the agent-relative
	prerogative. The agent-relative prerogative is a principle which states that
	an agent is permitted to weigh their own ends more heavily than another’s
	interests. The prerogative can also extend to other beings that an agent is
	responsible for or have strong relations with --- for example, a person is
	also permitted to give greater weight to the interests of their child. I
	will not make an extensive argument for it here; see Quong for a more
	detailed discussion.\footnote{maybe I should motivate it just a little more
	though}

	In a situation when at least one being’s rights will be violated (like an
	innocent attack), the agent-relative prerogative gives an agent the
	permission to prefer that another’s rights are violated instead of their
	own. This permits proportional self-defense against innocent attackers.
	
	\subsection{Proportionality and Property}

	The proportionality constraint of self-defense requires acts of defense
	to not cause much more harm than they seek to prevent. In paradigm cases
	of defensive force against a liable attacker, the attacker’s interests
	are weighed less heavily because they have made themselves liable. In
	addition, a defender (but not a third party) is permitted to weigh their
	own interests more heavily under the agent-relative prerogative.

	In contrast, in a case of defense against an innocent attacker, the
	attacker’s interests do not have a reduced strength. Instead, the defender’s
	agent-relative prerogative and increased weight to their own interests is
	the only justification for their defense. A third-party would not have
	reason to prefer the defender’s interests over the attacker’s.

	The fact that the defender’s rightful property is endangered is a threshold
	condition for the use of defensive force. The proportionality of the
	defensive force is determined in part by the strength of the defender’s
	interests that are at stake. This means that proportionality will look very
	different for different pieces of property. A piece of seaweed or a plot of
	farmland that is critical to one’s life or livelihood represents a strong
	interest. On the other hand, other rightfully possessed things may be less
	important and permit a lesser amount of defensive force to be used to
	preserve them.

\section{Non-Rational Animals and Non-Rational Humans}
 
	The previous section described a right of property and rightful defense
	of property against innocent, human attackers. In this section, I will
	describe some of the differences between innocent human attackers and
	nonrational animal attackers and why they are relevant to ethics.
    
    There are important differences between non-human and human animals.
    These differences are unrelated to the mental capacities of different
    beings. Instead, these differences have to do with the special relational
    duties we have to most non-rational humans and the differences in size and
    number that necessitate violent response against nonhumans in more cases
    than for humans.

	The intuitions highlighted by Mary Anne Warren pick out these differences
	rather than differences in the underlying duties we owe to humans and
	animals because of their basic capacities. I am not arguing that there are
	no differences in capacity-based duties towards humans and nonhumans.
	Instead, I am arguing that these differences are not necessary to explain
	differing intuitions about the permissibility of defensive
	force.\footnote{The argument structure here is similar to the section in
		Palmer that references Pogge’s work. The idea is to show that different
	intuitions about humans and animals can be explained without reference to
	moral status by pointing to all of the other different factors.}

    \subsection{Beneficence}

		In Clare Palmer’s \emph{Animal Ethics in Context}, Palmer argues for a
		limited version of the Laissez-Faire Intuition (LFI). Palmer’s version
		of the LFI holds that we have few, if any duties of beneficence to wild
		animals that are unaffected by human actions. However, we owe wild
		animals duties of beneficence when we benefit from institutions that
		harm them, when we form inter-special communities with
		them,\footnote{Garibaldi} or when we are causally responsible for
		unwarranted harm to them.

		In most circumstances, we have few obligations of beneficence to the
		rats, raccoons, and other liminal animals that largely benefit from
		human expansion and human spaces.

		The obligation of beneficence is related to self defense because it sets
		an upper bound on the proportionality of self-defense through the
		agent-relative prerogative. In paradigm cases of self-defense, the
		offending party has made themselves liable in some way when they choose
		to take a violent action.

		But the logic of self-defense through the agent-relative prerogative is
		different. When defending oneself against a non-culpable attacker, we do
		not assume that the attacker has forfeited or decreased the strength
		of their rights against being harmed. Instead, the defender is
		permitted to take harmful defensive action because they are permitted
		to weigh their own interests, goals, and projects more highly than that
		of the attacker. If they may prevent harm to themselves by redirecting
		it towards a third party (without \emph{using} that third party’s body
		or property in the process), then they are permitted to do so.

		Their obligation of beneficence to that third party sets an upper bound
		on the amount of harm they may redirect relative to the amount of harm
		they would accept. Let’s say that a defender must kill an innocent
		attacker in order to prevent the loss of a leg. This might very well
		be permissible under the agent-relative prerogative.

		If a defender would be duty-bound to accept the loss of a leg to (for
		example) save 5 innocent attackers, then they may not defend themselves
		against those innocent attackers for the sake of their leg. This does
		not imply that they may defend themselves against 3 attackers --- it is
		worse to cause harm than to fail to rescue a person from harm. However,
		the beneficence continues to serve as an upper limit, a maximum on the
		amount of harm that can be imposed to protect oneself.

		Beneficence can be more than just an upper limit. The degree of
		beneficence that we owe to a person is directly related to the strength
		of their interests compared to ours. If we have a strong duty of
		beneficence to a person (or animal), then we ought to meet their ends
		even at considerable personal cost. Their interests weigh heavily on
		the scale compared to our own. 

		It is a similar weighing of interests that determines proportionality
		for defense against innocent attackers. We are permitted to weigh our
		own interests more heavily than the equal interests of another. The
		weight we give the other being’s interests is directly related to the
		comparison underlying the obligation of beneficence. The main difference
		between defense against innocent attackers and the duty of beneficence
		is that the former case is concerned with doing harm and the latter
		with giving aid. In general, it is worse to do harm than to fail to
		give aid, so the scales are adjusted slightly differently in the case
		of innocent self defense. But the quantity measured (the difference in
		weight between the defender and the attacker’s interests) is the same
		in both cases.

		So, when we have a greater relational obligation of beneficence to
		someone, we also have greater duties to accept the harm they innocently
		impose in order to avoid harming them.

		Note that the general duty of beneficence is different from specific
		compensatory duties. My obligation to repair the damage that I caused
		to another person’s car doesn’t make me more liable to the innocent
		threats that they may impose. However, my general relational duties
		of beneficence towards a person (say, because they are my family member
		or coworker) give me a reason to weigh their interests more heavily both
		in my duties of beneficence and my prerogative of defense against
		innocent attackers.

    \subsection{Proxy Rationality}
		As Christine Korsgaard puts it, animal $+$ rational $\neq$ human.
		A non-rational human is (almost always) dependent on other humans
		to survive. Many non-rational animals are capable of securing their
		own ends without human help.

		Non-rational humans are generally in the care of other, rational
		humans. Human communities and sometimes individual humans usually
		have strong duties of beneficence to care for non-rational humans.
		When non-rational humans are in the care of rational humans, that
		human can take on some of the duties of ensuring that a non-rational
		human secures their ends in accordance with the relevant duties
		and laws that would apply if the non-rational human were rational.

		Rational humans are responsible for the actions of non-rational
		humans in their care. I don't mean this in the retrospective sense
		of culpability or moral responsibility. Rather, I mean it in the
		prospective sense. If a child steals a candy bar from a store, their
		parent is not morally responsible for the theft. However, the parent
		is responsible for making the store whole and educating their child,
		if appropriate.

		Parents or other guardians can also be responsible in the retrospective
		sense. A parent may be negligent, permitting their child to intrude on
		others' personal space. If so, that parent would be culpable for their
		negligence and responsible for apologizing to others and making them
		whole.  It is not always the case that the guardians of non-rational
		humans actually act in accordance with their duties. But the important
		thing is that those duties actually exist and have a person that ought
		to fulfill them.

		This is not true of many non-rational animals. These animals cannot
		fulfill duties\footnote{I think they can have duties though. Most humans 
		aren't rational most of the time, but we can enforce duties against
		them while they're asleep. Similarly, I can enforce a duty against an animal
		who would have had such a duty were they rational.} and do not have
		a 'proxy' rational agent that can take responsibility for their
		actions.

		The upshot of all this is that when there is some assurance that
		property rights will be respected or duly compensated, the importance
		of defending the right decreases. In addition, a guardian's
		responsibility can provide some assurance that future trespasses do not
		occur. We can infer that a rat who is expelled may intrude again if
		they are not left far away, but we cannot infer the same when we return
		a non-rational human (or animal) to their guardian.\footnote{I don't
		think I'm being incredibly clear in this passage whether the existence
		of the duty is important or whether the pragmatic assurance is important.
		I'm not too clear on that myself.}

	\subsection{Size, Number, and Future Expectations}

	There are a few pragmatic conerns that makes violence against non-human
	invaders necessary more often than violence against human invaders.
	Non-human animal invaders are often smaller than humans. There are also more
	of them. In common cases, they also multiply --- a few rats in a home today
	may mean a dozen in a few months.

	The first two points have to do with the necessity constraint of
	self-defense.  In cases of self defense, no person deserves to be harmed.
	Instead, costs are distributed in the fairest way possible. However,
	overall costs should still be minimized. If a defender can take choose
	between a less harmful defense and a more harmful defense (both of which
	fully address the threat), they have an obligation to choose the former.
	That obligation has nothing to do with the degree of status or
	consideration given to the attacker.

	Because humans often occupy spaces designed for humans, it is often
	possible to address innocent human invaders using non-lethal means.
	However, non-human invaders occupy spaces in our walls and crevices in such
	a way that it is more difficult to find them and more difficult to remove
	them non-lethally. This doesn’t mean that they have fewer moral rights ---
	if it were possible to do so, there would still be an obligation to take
	the least harmful means. Instead, it just means that in some cases, it
	might be permissible to kill a nonhuman animal where a human threat could
	be addressed without violence.

	The final point --- the fact that contramensal animals often multiply ---
	has to do with the magnitude of the threat. The presence of several
	termites in the walls does not necessarily weigh greatly on the interests
	of the humans who live there. However, a few termites will likely become a
	few thousand. So, the anticipated threat that the termites present is not
	just their presence in the wall, but in the likely future harm that they
	might impose.

	The likely possibility of escalating harm permits a wider scope of
	defensive actions actions now in order to avoid future harms in addition to
	imminent harms.

	\subsection{The Role of Moral Status}

	The three non-capacity characteristics that humans often have (stronger
	duties of beneficence, proxy rationality, and pragmatic differences)
	explains most of the difference between defense against, say, rats and
	defense against, say, children.  Companion animals and domestic animals
	have all three of these qualities.  Defenders have stronger duties of
	beneficence, the animals are in the care and control of a guardian, and
	they are few in number, large in size, and slow in reproduction. When we
	compare dogs and children, I think the gap in our intuitions about the
	morality of self defense certainly shrinks.

	There is still a gap --- and maybe that gap is attributable to a difference
	in relative moral status. But that does relatively light lifting compared to
	these other features. It also only ‘kicks in’ in situations when we must
	compare the interests of humans and animals anyway (like innocent
	self-defense or lifeboat scenarios). It does not permit the treatment of
	animals as if they have no ends at all.

	It certainly doesn’t do to what Warren, DeGrazia, and others want it to do
	in permitting animal research or hunting.

\section{Animals as Property Holders}

	I think that animals clearly mark and claim property as well.
	\texttt{Cheryl Abbate Liability to defensive harm.} Humans become liable to
	defensive harm when they intrude on animal property.  If I poke a beehive,
	I am perfectly liable for the stings that I receive afterwards, and I
	shouldn’t use defensive force against the bees.  Same goes for poking
	around in bare caves and sticking a hand down a snakehole.

\end{document}
