\documentclass[12pt]{article}
\usepackage{mathpazo}
\usepackage[backend=bibtex]{biblatex-chicago}
\bibliography{../thesis}

\begin{document}

\section{Animals and Property}

	Defense of property sometimes justifies actions that are otherwise
	impermissible. If a person is walking across the street in front of us, we
	may not demand that they leave. We certainly should not harm them or treat
	them as liable to harms.  We can certainly demand that a person walking
	around our home leave. If they ignore the demand, they may be liable to
	harm.

	These statements are relatively uncontroversial when referring to rational
	persons who willingly and knowingly trespass on private, individually owned
	or occupied property. This becomes less clear when discussing nonhuman
	animals. Few of them could be described as rational and even fewer might
	understand the concept of property. Even if they could, they might fall
	outside the covenant or agreement that protects property rights.

	However, it seems clear that the concept of self-defense or defense of
	property can still be applied against animals.  If there are insects living
	in my skin, I feel no qualms about killing them, even though they may not
	be aware they are invading my body.\footnote{I will assume for the moment
	that these insects should be accorded moral status though it is unclear.}
	I am also willing to believe that it is easier to justify capturing or even
	killing rats in the home rather than rats living in the woods.

	Property, or something like it, is a pre-theoretical and pre-human notion.
	I would describe a beehive as the property of the bees who live in it.
	Marked territory seems like a kind of property that a wolf seeks to claim.
	And I am certainly unlikely to intrude on a bear in their hibernation den.
	\texttt{TODO citations/explanations from Wildlife as Property Owners}.  If
	humans claim special protections and a right to defend their own property,
	then I don’t think that it’s unreasonable to think that animals might have
	similar, special rights to their own as well.

	In this section, I’ll consider the nature of the right of property, the
	extent to which a defense of property can be applied against non-human
	animals, proportionality in defending property, and whether animal
	territories can be described as their property.

	\subsection{The Nature of Property}

	Property are the things that a being controls in order to secure their
	ends.  To hold a resource as property is to have a rightful claim to use it
	for any purpose that does not interfere with the choice or similar right of
	another.  To claim an unclaimed resource as property, one need only
	empirically control it and signal an intent to continue controlling it.
	The concept of property is not dependent on the existence of an agreement to
	enforce it.

	Consider a pared down example. A table is laid out with a collection of
	books on it, along with a sign saying “Free to Take.” I see a book I like,
	and I intend to collect it after class. If another person takes the book
	before I do, then they have not wronged me.

	On the other hand, imagine I actually did take the book. Another person
	sees it in my bag and chooses to take it from me. This person has wronged
	me because they used an object which I had marked as my property in a way
	that I did not authorize.

	This example also demonstrates that an interest-based theory of property
	alone is insufficient. In both cases, my interests are set back by equal
	amounts. However, I am only wronged in the latter case.

	I cannot merely claim that all the books on the table are mine because I do
	not control them. If I were to scoop them all up into a cart and claim them
	as my own, I would be acting wrongfully. However, the wrong would not be
	a violation of any individual person’s right to property. Rather, it would
	be a selfish act and wrong for that reason, or perhaps violate the implicit
	conditions on claiming attached to the books by the original owner.

	Let’s pare the example back further and assume that the only resources in
	the world were these books. If this were the case, a person who claimed all
	of the books would give other people no reason to respect a putative right
	to property. They cannot expect their right to pursue their ends to be
	respected when they cannot accord the same respect to others --- after all,
	there are no property lines that can be respected.

	Even within a state of nature, prior to any unification of will or
	hypothetical agreements, intruding upon a person’s rightfully claimed
	personal property renders the intruder liable to defensive harm unless the
	property owner is duty-bound to accept the intrusion.

	Consider a desert island with no human or nonhuman animal inhabitants.
	A shipwrecked sailor washes up to shore. Her leg is wounded. In order to
	save a limb, she can use a special piece of seaweed that has washed up on
	shore with her. She uses the seaweed to cover the wound. Soon after, another
	person washes up alongside. The second person also requires the seaweed to
	staunch a wound. The second person has an interest in the seaweed that is
	equal to the interest of the first person. It seems clear here that the
	second person could not take the seaweed from the first.

	Even if the first person temporarily sets the seaweed aside, a signal and
	an intent to continue using it seems to demarcate her property. The second
	person ought not take the seaweed from the first. If the second person does
	so, then they render themselves liable to proportional defensive harm from
	the first.

	\subsection{Animal Claims}

	I think that animals clearly mark and claim property as well.
	\texttt{Cheryl Abbate Liability to defensive harm.} Humans become liable to
	defensive harm when they intrude on animal property.  If I poke a beehive,
	I am perfectly liable for the stings that I receive afterwards, and I
	shouldn’t use defensive force against the bees.  Same goes for poking
	around in bare caves and sticking a hand down a snakehole.

	\subsection{The Agent-Relative Prerogative}

	In this subsection, I’ll briefly sketch out Jonathan Quong’s agent-relative
	prerogative and its application to the defense of property.

	\subsection{Defense of Property Against Non-Rational Nonhumans}
	
	Through the agent-relative prerogative, the principles described above
	continue to apply to non-responsible beings, humans and nonhumans alike.
	A person’s rightfully claimed property can use proportional defense to
	preserve their property. 

	\subsection{Proportionality}
	\subsection{Animals as Property}

\end{document}
