\documentclass[12pt]{book}
\usepackage{mathpazo}
\usepackage[backend=bibtex]{biblatex-chicago}
% \bibliography{../thesis}

\begin{document}

\chapter{Animals and Property}

    In her work \emph{Moral Status}, Mary Anne Warren argues that animals have
    a lower moral status than humans.\footnote{Warren is a little dated...
    should I be bringing her up here?} Warren's argument relies on the intuition
    that we are willing to kill rats that invade our homes where we would not
    be willing to kill human children if they were in the same position.
    
    Warren's argument has already been rejected\footnote{that one 2007 paper},
    but her hypothetical raises unanswered questions. Human and animal interests
    come into constant conflict over the Earth's land and resources. It seems as
    if any hard-line animal rights theory would be forced into an uncomfortable
    conclusion. To minimize killing animals over land or resource
    conflicts, humans would be forced to return to gatherer lifestyles.
    
    In the previous section, I concluded that it is impermissible to use
    animals as mere resources for human ends. In this section, I will consider
    the duties we have to share the resources of the Earth with other animals.

	Property, or something like it, is a pre-theoretical and pre-human notion.
	A beehive as the property of the bees who live in it.  Marked territory
	seems like a kind of property that a wolf seeks to claim.  And I am
	certainly unlikely to intrude on a bear in their hibernation den.
	\texttt{TODO citations/explanations from Wildlife as Property Owners}.  If
	humans claim special protections and a right to defend their own property,
	then I don’t think that it’s unreasonable to think that animals might have
	similar, special rights to their own as well.
    
    I will conclude that humans and animals both have subject-relative 
    prerogatives to defend their rightfully acquired property. This prerogative
    permits the use of lethal defensive force as a final resort to defend
    a person's closely held property against nonhuman and human invaders alike.
    However, there are circumstances under which humans may also be liable
    to defensive force or property intrusions.
    
    Furthermore, I will argue that humans do not have a duty to re-wild
    long-held human spaces for the sake of other animals, but that all people
    have strong duties to reduce or eliminate the harm caused through
    continuing human expansion.
    
\section{Defense of Property}

	Defense of property sometimes justifies actions that are otherwise
	impermissible. If a person is walking across the street in front of us, we
	may not demand that they leave. We certainly should not harm them or treat
	them as liable to harms.  We can certainly demand that a person walking
	around our home leave. If they ignore the demand, they may be liable to
	harm.

	These statements are relatively uncontroversial when referring to rational
	persons who willingly and knowingly trespass on private, individually owned
	or occupied property. This becomes less clear when discussing nonhuman
	animals. Few of them could be described as rational and even fewer might
	understand the concept of property. Even if they could, they might fall
	outside the covenant or agreement that protects property rights.

	However, it seems clear that the concept of self-defense or defense of
	property can still be applied against animals.  If there are insects living
	in my skin, I feel no qualms about killing them, even though they may not
	be aware they are invading my body.\footnote{I will assume for the moment
	that these insects should be accorded moral status though it is unclear.}
	I am also willing to believe that it is easier to justify capturing or even
	killing rats in the home rather than rats living in the woods.

	In this section, I’ll consider the nature of the right of property and the
	morality of defense of property. In the following section, I will expand
	on this baseline and apply the 

	\subsection{The Nature of Property}

	Property are the things that a being controls in order to secure their
	ends.  To hold a resource as property is to have a rightful claim to use it
	for any purpose that does not interfere with the choice or similar right of
	another.  To claim an unclaimed resource as property, one need only
	empirically control it and signal an intent to continue controlling it.
	The concept of property is not dependent on the existence of an agreement to
	enforce it.

	Consider a pared down example. A table is laid out with a collection of
	books on it, along with a sign saying “Free to Take.” I see a book I like,
	and I intend to collect it after class. If another person takes the book
	before I do, then they have not wronged me.

	On the other hand, imagine I actually did take the book. Another person
	sees it in my bag and chooses to take it from me. This person has wronged
	me because they used an object which I had marked as my property in a way
	that I did not authorize.

	This example also demonstrates that an interest-based theory of property
	alone is insufficient. In both cases, my interests are set back by equal
	amounts. However, I am only wronged in the latter case.

	I cannot merely claim that all the books on the table are mine because I do
	not control them. If I were to scoop them all up into a cart and claim them
	as my own, I would be acting wrongfully. However, the wrong would not be
	a violation of any individual person’s right to property. Rather, it would
	be a selfish act and wrong for that reason, or perhaps violate the implicit
	conditions on claiming attached to the books by the original owner.

	Let’s pare the example back further and assume that the only resources in
	the world were these books. If this were the case, a person who claimed all
	of the books would give other people no reason to respect a putative right
	to property. They cannot expect their right to pursue their ends to be
	respected when they cannot accord the same respect to others --- after all,
	there are no property lines that can be respected.

	Even within a state of nature, prior to any unification of will or
	hypothetical agreements, intruding upon a person’s rightfully claimed
	personal property renders the intruder liable to defensive harm unless the
	property owner is duty-bound to accept the intrusion.

	Consider a desert island with no human or nonhuman animal inhabitants.
	A shipwrecked sailor washes up to shore. Her leg is wounded. In order to
	save a limb, she can use a special piece of seaweed that has washed up on
	shore with her. She uses the seaweed to cover the wound. Soon after, another
	person washes up alongside. The second person also requires the seaweed to
	staunch a wound. The second person has an interest in the seaweed that is
	equal to the interest of the first person. It seems clear here that the
	second person could not take the seaweed from the first.

	Even if the first person temporarily sets the seaweed aside, a signal and
	an intent to continue using it seems to demarcate her property. The second
	person ought not take the seaweed from the first. If the second person does
	so, then they render themselves liable to proportional defensive harm from
	the first.

\section{Non-Rational Animals and Non-Rational Humans}
    
    Mary Anne Warren's comparison of our duties to
    animals with our duties to non-rational humans is common in animal
    ethics. These comparisons have been criticized from both
    sides of the issue, but there are specific reasons why these comparisons
    muddle intuitions about the defense of property.
    
    There are important differences between non-human and human animals.
    These differences are unrelated to the mental capacities of different
    beings. Instead, these differences have to do with the special relational
    duties we have to most non-rational humans and the differences in size and
    number that necessitate violent response against nonhumans in more cases
    than for humans.

	The intuitions highlighted by Mary Anne Warren pick out these differences
	rather than differences in the underlying duties we owe to humans and
	animals because of their basic capacities. I am not arguing that there are
	no differences in capacity-based duties towards humans and nonhumans.
	Instead, I am arguing that these differences are not necessary to explain
	differing intuitions about the permissibility of defensive
	force.\footnote{The argument structure here is similar to the section in
		Palmer that references Pogge’s work. The idea is to show that different
	intuitions about humans and animals can be explained without reference to
	moral status by pointing to all of the other different factors.}

    \subsection{Beneficence}

		In Clare Palmer’s \emph{Animal Ethics in Context}, Palmer argues for a
		limited version of the Laissez-Faire Intuition (LFI). Palmer’s version
		of the LFI holds that we have few, if any duties of beneficence to wild
		animals that are unaffected by human actions. However, we owe wild
		animals duties of beneficence when we benefit from institutions that
		harm them, when we form inter-special communities with
		them,\footnote{Garibaldi} or when we are causally responsible for
		unwarranted harm to them.

		In most circumstances, we have few obligations of beneficence to the
		rats, raccoons, and other liminal animals that largely benefit from
		human expansion and human spaces.

		The obligation of beneficence is related to self defense because it sets
		an upper bound on the proportionality of self-defense through the
		agent-relative prerogative. In paradigm cases of self-defense, the
		offending party has made themselves liable in some way when they choose
		to take a violent action.

		But the logic of self-defense through the agent-relative prerogative is
		different. When defending oneself against a non-culpable attacker, we do
		not assume that the attacker has forfeited or decreased the strength
		of their rights against being harmed. Instead, the defender is
		permitted to take harmful defensive action because they are permitted
		to weigh their own interests, goals, and projects more highly than that
		of the attacker. If they may prevent harm to themselves by redirecting
		it towards a third party (without \emph{using} that third party’s body
		or property in the process), then they are permitted to do so.

		Their obligation of beneficence to that third party sets an upper bound
		on the amount of harm they may redirect relative to the amount of harm
		they would accept. Let’s say that a defender must kill an innocent
		attacker in order to prevent the loss of a leg. This might very well
		be permissible under the agent-relative prerogative.

		If a defender would be duty-bound to accept the loss of a leg to (for
		example) save 5 innocent attackers, then they may not defend themselves
		against those innocent attackers for the sake of their leg. This does
		not imply that they may defend themselves against 3 attackers --- it is
		worse to cause harm than to fail to rescue a person from harm. However,
		the beneficence continues to serve as an upper limit, a maximum on the
		amount of harm that can be imposed to protect oneself.

		Beneficence can be more than just an upper limit. The degree of
		beneficence that we owe to a person is directly related to the strength
		of their interests compared to ours. If we have a strong duty of
		beneficence to a person (or animal), then we ought to meet their ends
		even at considerable personal cost. Their interests weigh heavily on
		the scale compared to our own. 

		It is a similar weighing of interests that determines proportionality
		for defense against innocent attackers. We are permitted to weigh our
		own interests more heavily than the equal interests of another. The
		weight we give the other being’s interests is directly related to the
		comparison underlying the obligation of beneficence. The main difference
		between defense against innocent attackers and the duty of beneficence
		is that the former case is concerned with doing harm and the latter
		with giving aid. In general, it is worse to do harm than to fail to
		give aid, so the scales are adjusted slightly differently in the case
		of innocent self defense. But the quantity measured (the difference in
		weight between the defender and the attacker’s interests) is the same
		in both cases.

		So, when we have a greater relational obligation of beneficence to
		someone, we also have greater duties to accept the harm they innocently
		impose in order to avoid harming them.

		Note that the general duty of beneficence is different from specific
		compensatory duties. My obligation to repair the damage that I caused
		to another person’s car doesn’t make me more liable to the innocent
		threats that they may impose. However, my general relational duties
		of beneficence towards a person (say, because they are my family member
		or coworker) give me a reason to weigh their interests more heavily both
		in my duties of beneficence and my prerogative of defense against
		innocent attackers.

    \subsection{Proxy Rationality}
		As Christine Korsgaard puts it, animal $+$ rational $\neq$ human.
		A non-rational human is (almost always) dependent on other humans
		to survive. Many non-rational animals are capable of securing their
		own ends without human help.

		Non-rational humans are generally in the care of other, rational
		humans. Human communities and sometimes individual humans usually
		have strong duties of beneficence to care for non-rational humans.
		When non-rational humans are in the care of rational humans, that
		human can take on some of the duties of ensuring that a non-rational
		human secures their ends in accordance with the relevant duties
		and laws that would apply if the non-rational human were rational.

		Rational humans are responsible for the actions of non-rational
		humans in their care. I don't mean this in the retrospective sense
		of culpability or moral responsibility. Rather, I mean it in the
		prospective sense. If a child steals a candy bar from a store, their
		parent is not morally responsible for the theft. However, the parent
		is responsible for making the store whole and educating their child,
		if appropriate.

		Parents or other guardians can also be responsible in the retrospective
		sense. A parent may be negligent, permitting their child to intrude on
		others' personal space. If so, that parent would be culpable for their
		negligence and responsible for apologizing to others and making them
		whole.  It is not always the case that the guardians of non-rational
		humans actually act in accordance with their duties. But the important
		thing is that those duties actually exist and have a person that ought
		to fulfill them.

		This is not true of many non-rational animals. These animals cannot
		fulfill duties\footnote{I think they can have duties though. Most humans 
		aren't rational most of the time, but we can enforce duties against
		them while they're asleep. Similarly, I can enforce a duty against an animal
		who would have had such a duty were they rational.} and do not have
		a 'proxy' rational agent that can take responsibility for their
		actions.

		The upshot of all this is that when there is some assurance that
		property rights will be respected or duly compensated, the importance
		of defending the right decreases. In addition, a guardian's
		responsibility can provide some assurance that future trespasses do not
		occur. We can infer that a rat who is expelled may intrude again if
		they are not left far away, but we cannot infer the same when we return
		a non-rational human (or animal) to their guardian.\footnote{I don't
		think I'm being incredibly clear in this passage whether the existence
		of the duty is important or whether the pragmatic assurance is important.
		I'm not too clear on that myself.}


\subsection{Animals as Property Holders}

	I think that animals clearly mark and claim property as well.
	\texttt{Cheryl Abbate Liability to defensive harm.} Humans become liable to
	defensive harm when they intrude on animal property.  If I poke a beehive,
	I am perfectly liable for the stings that I receive afterwards, and I
	shouldn’t use defensive force against the bees.  Same goes for poking
	around in bare caves and sticking a hand down a snakehole.

\end{document}
