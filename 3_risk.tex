\chapter{Risk of Harm to Non-human Animals}

Risks of harm pervade our daily lives. Driving cars, operating oil refineries,
and even flipping light switches impose risks on other beings.  The moral
boundaries of imposing risk are hardly settled even in the case of humans.
However, the issue has yet to be directly addressed in the case of non-human
naimals, even though it has a tremendous significance in environmental and
public policy.

In this part, I will briefly sketch a theory of risk imposition that is
informed by the basic deontic views I have accepted in previous parts.  I
will then identify three elements of the ethics of risk imposition that are
unique to nonuhuman animals.

First, I’ll address the intuition that it is worse to impose risk on wildlife
than it is to impose risk on contact-zone or liminal animals.  Second, I will
discuss the relationship between human duties to avoid imposing risk and the
expansion of animal population.  Third, I will consider the problem of
vultures, or the moral value of secondary benefits generated through
potentially wrongful risk imposition.

\section{A General View of Risk}

Discussions of the moral permissibility of risk can be divided into two
different categories: the duties of an individual risk-imposer, or the duties
of a regulator setting rules for individual risk imposition. For the purposes
of this paper, I will collapse the two discussions by setting aside practical
considerations about the implementation or enforcement of policy. What I am
interested in is the values that regulators and individuals alike should seek
to uphold.

\subsection{Cost-Benefit Analysis}

The dominant view in modern U.S. regulation is cost-benefit analysis, a close
relative of expected utility theory. Regulators use economic methods to
estimate the cost that a person is willing to pay to avoid a particular amount
of risk.  This measure is used to ‘price’ a human life, providing a guidepost
to determine the costs that must be paid in order to avoid risky actions. This
measure is often referred to as the Willingness To Pay (WTP) metric.

This view is defended by Cass Sunstein, who argues in favor of its efficacy and
that this measure is consistent with consequential and deontological theories.
It seeks to maximize expected consequences and also treats people in accordance
with the rules they create for themselves, thus respecting their
dignity.\autocite{sunstein_cost_benefit}

Apart from possible methodological shortcomings,\footnote{statlife} this view is
most closely criticized by Lisa Heirtzerlinger, who argues that Sunstein
essentially assumes that risk impositions are de facto permissible and that
safety precautions are sometimes required. Sunstein does not ever consider that
the risky activity as a whole might be impermissible, or consider what the
justification for engaging in the activity might be.

In addition to Heirzerlinger’s criticism, Sunstein’s argument is also
vulnerable to the objection that the methodology of WTP does not capture what
it is permissible for \emph{other} people to do to me. I may permissibly choose
to save \$50 of my dollars by accepting some small risk. However, this is not
evidence that I would endorse the rule that some other person could benefit to
the tune of \$50 while exposing \emph{me} to the same
risk.\autocite[98]{hansson_risk} While Sunstein argues that production costs
will eventually be passed down to the consumer, Sunstein does not consider that
the person at risk and the person benefiting could be different --- a factory
could impose risk of harm to people around area A, but produce products which
are primarily sold in area B. In this situation, the residents of area B are
benefiting from the loss of residents of A with no clear economic mechanism to
correct this inequity.

A more fundamental concern with the WTP approach is that it violates some of
the core commitments of deontological ethics; namely, that it assigns a price
to that which is priceless --- a life. Though Sunstein and advocates of WTP
argue that their practice does not assign a price to life, and merely a cost
for some statistical benefits. Despite this rhetorical shift, the WTP theorist
would still be acting as though they had assigned a fixed price to
dignity.\autocite{kant_uncertainty}

The issue with the rhetorical shift is further revealed by Lisa Heirtzerlinger,
who enumerates a number of uncomfortable consequences of the WTP metric that
really seem like instances of valuing dignity itself, despite what the
economists might respond.

Importantly, the cost-benefit analysis view reduces the value of animals and
the environment to the instrumental costs that people are willing to pay to
protect them. Non-human animals generally do not actively participate in the
human economy, so we cannot determine a WTP metric for them, no matter what
normative force it would carry. This use of the WTP metric treats the
environment and other animals as merely instrumental and only worth the price
that humans are willing to pay. The single strongest argument in favor of the
WTP metric is that it describes self-authored preferences regarding risk, and
in a way respects the agency of a person. However, this argument does not apply
to the way that WTP is used to price other animals and the environment.

Even if cost-benefit analyses are the most efficient way to regulate and
control our environmental impact in an increasingly complex economy, because of
the preceding problems, cost-benefit analyses do not have great normative force
and cannot serve as the guideposts of our moral values.  Instead, it must be
the other way around --- we should appraise the moral values we hold and
perhaps accept cost-benefit analyses as the lesser of many evil means to best
uphold those values.

\subsection{Risks as Reciprocal, Connected Practices}

Sven Ove Hanson’s view of risk as a set of connected social practices could
provide a moral foundation for our imposition of risk on our fellow creatures.
Hanson argues for a ‘fair exchange of risk’ model, concluding with the principle
that:

\begin{quote}
  Exposure of a person to risk is acceptable if (i) this exposure is part of a
  persistently justice-seeking social practice of risk-taking that works to her
  advantage and which she de facto accepts by making use of its advantages, and
  (ii) she has as much influence over her risk-exposure as every similarly
  exposed person can have without the loss of the benefits that justify that
  exposure.\autocite[107]{hanson}
\end{quote}

This principle is agreeable, and Hanson argues for it capably. It provides an
ethical basis for many day-to-day practices that essentially everyone would
like to accept as permissible.  It also outlines the reasons that  we can call
another to account for their risk taking.  We might argue that imposed risk is
unreasonable (it goes beyond common social practice); that it is unjust (it is
unevenly concentrated); or that it is unaccepted (the relevant class of people
exposed to risk have not had the opportunity to accept it, or have explicitly
rejected it).

Hanson’s view is also consistent with the intuition that, all things being
equal, it is worse to impose risk on people outside of community that benefits
from the risk imposition than it is to impose risk on people inside of that
community. In other words, it would be worse to divert a U.S. factory’s
pollution to a town in Canada than towards a town on this side of the border.
This acknowledges the political dimension of risk and the role that consent and
appropriate procedure play in risk imposition.

One point that Hanson does not make about reciprocal risk exchange but
ultimately bolsters his stance is that it may not actually be necessary for a
person to exercise their liberty to take risks under a social agreement for
them to benefited by that liberty. In other words, if my neighbor and I both
have equal say in a rule that permits everyone to use a driving lawnmower, it
is not necessarily unfair or unequal to me if I do not use a driving lawnmower.
I retain the option to use one of I choose; I am held to the same rules as my
neighbor even if I do not seek to exercise them to the fullest extent.

Of course, if the rules are structured to permit all of my neighbor’s desires
and none of mine, I might claim that the rules are unjust. In practice,
policies that are in place seem to exhibit a preference to permit risky
activities of the upper classes when equally risky behaviors of the
impoverished are condemned.\footnote{\emph{See e.g.,} \cite[Ch. 1]{pl}}

\subsection{“Social Practice”}

One ambiguity in Hanson’s view is how widely to draw the net of ‘social
practice.’ Does social practice include any and all socially-acceptable risky
acts? We might accept that driving is a social practice, whereas archery is
not, as Jonathan Quong argues.\autocite[35,50]{quong} Or, we might choose a
broader view that collects and considers essentially all permitted social
practices as one big social exchange.

I don’t think that there are moral reasons to distinguish between activity
types in this way. As Helen Frowe argues, if a risky action earns the actor
some benefit at the cost of some risk of harm to others, the class of the
action does not seem to affect its morality.\autocite[9]{frowe}

It may be legal to drive around for enjoyment but not shoot guns up in the air
for enjoyment, even if both impose the same risk to others. But I think that
this is a convenience of legislation and not a moral implication. We all agree
to some level of risk that we tolerate in order to benefit from those
impositions and impose risks of our own, but this exchange doesn’t speak to the
class or type of token actions of risk so long as they have acceptable costs
and sufficient benefits.

So, when Hanson discusses the social practice of risk-taking, he must refer to
it in the broadest sense; all risky acts which, taken together, leave us better
off than the absence of all these acts.

This seems to leave Hanson with a problem. Consider some system of risk that
left people slightly better off than no risks at all, but was much worse than
many other possible systems. Hanson’s view seems to allow this is a permissible
condition. However, Hanson could respond to this by appealing to the
‘justice-seeking’ element of his stance. The sub-optimal system of risk
described above could only be permissible if those who controlled it were
working to achieve a more optimal system. Without sufficient effort to seek
improvements to the system, the system’s continued existence and imposition of
risk would be wrongful.

\section{Animals and Reciprocal Risks}

Hanson does not consider risks imposed to non-human animals. However, I believe
that non-human animals fit well into this framework.

In this section, I am most interested in non-domesticated liminal and wild
animals. Domesticated animals fit clearly into Hanson’s framework as wards of
people that do buy into systems of risk and have representation in the way that
risks are permitted.

Hanson’s view requires that (1) beings accept benefits from risky practices,
(2) beings have influence over the risky practices that are conducted, and (3)
that the risky practices are justice-seeking, aiming to equalize costs and
benefits.  On first glance, it seems as if non-human animals cannot possibly
fit this framework because non-human animals do not themselves engage in
significantly risky practices towards human life.

Though nonhumans do not themselves generate significant risks, many of them do
benefit greatly from the risky practices that constitute urban life. The cars
that rumble by bring home an important source of food for racoons. The factory
that produces french fries will benefit the pigeons that feast on leftovers in
a parking lot.\autocite[68]{zoopolis} Many liminal animals thrive among the risks
or urban life and derive benefits from the processes that generate that risk.
Not least among the benefits is the exclusion of less-adapted competitors that
would otherwise fight for the same resources.

I want to be careful here in not referencing the abstract idea of ‘animals’ or
‘the animal.’ I am not saying that the class of liminal animals benefits, or
that the class of racoons benefit. Classes of beings do not ‘benefit,’ at least
not in the way that biotic ethicists mean. I do not mean that urban life allows
more pigeons to exist than the absence of urban life, or that ‘the pigeon’
flourishes more in urban life. If I were to make these claims, I would butt up
against a non-identity problem because different species and kinds of animals
would exist without an urban landscape. Instead, what I mean is that most
\emph{individual} pigeons eat food that would not be available but for a complex
web of exchanged risks and make nests in buildings that would not be there but
for social practices of risking.

An important benefit of this view is that it explains the intuition that it is
worse to expose non-contact zone animals to risks than it is to expose liminal
or contact-zone animals to risk. This is because the latter are
much more likely to benefit from human social practices of risking. This
intuition is seen in, for example, Donaldson \& Kymlicka’s mention of
animal bypasses over remote roads, but neglect of the possibility of ‘squirrel
bridges’ over town streets.\footnote[167]{zoopolis} This is not to say that squirrel
bridges should not be built, or even that they are not obligatory. Instead, my
conclusion is that harm to creatures that do not benefit from human practices
of risking is more morally bad than harm to creatures that do benefit.

\subsection{Objections}

There are two primary objections that can be made against including animals
within reciprocal risk relationships in this way. The benefits that human
practices of risk generate for the other animals is merely incidental. This
undercuts the idea that animals can be said to be coequal beneficiaries of a
risky system.  The second is that there is no necessary condition between the
imposition of risk on these animals and the benefits they garner. It would be
possible for people to provide the same benefits to other animals with far
fewer risks.

The primary problem with both of these objections is that they are both equally
applicable to the case of humans. Social practices of risking are not made to
benefit any specific person; in a way, any benefit they generate for an
individual are merely side effects of the churning of a much larger system. In
addition, it is not clear that the \emph{intention} underlying the benefit is
morally significant (even if an intention can be ascribed to the social
practice of risk, which no single agent authorizes or enacts). It is not by the
benevolence of the butcher or the baker that a person is able to enjoy their
seitan and bread, but they are certainly still deriving a benefit.\footnote{You
could say at least that the butcher and baker intend to serve the person even
if not benevolently... the rats that eat the baker’s scraps may not owe her
anything. Something to consider.}

The second objection could also apply to humans --- no person actually accepts
or desires to take on a specific risk. In many circumstances, it would be
possible for a society to choose to protect a person from all risks while still
providing them with all of the benefits of society. However, this is just an
elevation of one person above all others; the system of risk-taking as a whole
is necessary to enjoy the benefits in general whether we choose to insulate one
person or not. In this case, we insult the dignity of most people by holding
that one otherwise co-equal beneficiary should bear no costs at all.

Similarly, we could theoretically limit all of the risk that we impose upon an
animal while still working to provide them the same benefits. But this would be
elevating that animal above all others and above humans. Limiting all risk
would result in dramatic changes to the urban and sub-urban landscape that they
benefit from.

\subsection{Representation}

The one piece of the puzzle that is currently missing are the justice-seeking
and co-representation elements of the reciprocal risk principle. Currently,
individual, non-endangered or migratory animals are essentially excluded from
direct consideration by regulatory agencies and legislatures that determine
acceptable exposures of risk.

This is wrong for two reasons. First, lack of representation of interests of
beings that have a real stake in rulemaking is wrong and causes impositions of
risk under those rules to be wrong. Second, pragmatically, the scales are
tipped against just distribution of costs and benefits because of this lack of
representation.

This may be remedied by appointing representatives of non-human animals on
road, emissions, and other environmental safety organizations. This may seem to
contradict the role of these animals (as Donaldson and Kymlicka would put it)
as denizens, not co-equal citizens. However, there are circumstances when it is
clear that denizens should be given input --- maybe not on the political
community’s core values and aims, but certainly on specific items that are
particularly relevant to their well being and reflect the basic duties of the
political community towards them. For example, migrant workers who are most at
risk of illness due to pesticide use should have a say (if not a
near-dispositive say) in the way that pesticides are applied and regulated.

Though denizens are not of the political community in the same way as citizens,
they share in the social practices of risking and have the right to
representation in the bodies that govern this practice.

\subsection{Restitution}

Imposing risks on a person can obligate restitution in two ways. First, one may
have a duty to compensate for imposing a risk as such. Second, one may have a
duty to hold another harmless through restitution for an injury that results
from a materialized risk.

A duty to provide restitution in these two cases has been described previously
by several different works.\footnote{\emph{See e.g.} \cite{hansson_risk},
\cite{frowe}, \cite{finkelstein}} But it is less clear how restitution might
work in the case of nonhuman animals. There are three primary difficulties in the
nonhuman case.  First, it is unclear whom restitution should go to. Second, it
is unclear what form restitution should take, since bank accounts are
meaningless to squirrels. Third, the amount of restitution that is morally
required is unclear, as non-human animals do not participate in the economic
system that is normally used as measuring tool for damages.

\subsubsection{To Whom}

In human cases, it is relatively clear who should be compensated for risky
activity. When compensating for risks as such, the class of people who are
believed to be at risk before the fact should receive some compensation for the
risk that is imposed on them. As Hansson notes, this compensation may not be
necessary if risks are fairly distributed or exchanged across different social
and productive sectors.\autocite[103]{hansson_risk} When compensating for
materialized injury, compensation should be given to the person that has
suffered an injury that is potentially caused by a human-imposed
risk.\autocite[112--113]{hansson_risk}


The first kind of compensation seems relatively clear for the case of non-human
animals. It may take extra work, but one could identify a class of non-humans
that are being exposed to additional risks. The second kind is more difficult.
In human cases, the tort system would likely depend on the person who is
injured to come forth with a claim against risk imposers. It is possible that
we may never know or find the individual animals who are injured as a result of
imposed risks.

But just as in human cases, there are some circumstances when we do not know
who exactly has been injured by an imposed risk. Hanson’s solution to this
(which can be extended to non-humans as well) is to compensate the class of
people who may be injured proportional to the magnitude of their injury and the
chance that they are injured. If we truly have no further information about the
class, then this may look exactly like antecedent compensation.

We do have some information about which animals are injured, however. People
often bring oiled, struck, or injured animals to wildlife hospitals. It is
among the compensatory duties of risk for car-drivers to contribute to wildlife
hospitals that care for wildlife injured by human-imposed risks.

If animals die because of imposed risk, compensation may be directed as it is
for dead humans --- towards the family and community that might suffer from the
loss.

\subsubsection{What Form}

Money in a bank account means nothing to squirrels, and squirrels cannot
themselves use it to accomplish their ends. I think this question is the
easiest to answer. Karen Bradshaw argues that the existing legal instrument of
trusts serve as a way for animals to hold and disburse property. Trusts are
legal persons like corporations or nonprofits, but they are created for the
sake of specific benefactors, and the executor of the trust is legally required
to use funds in a way that most benefits the trustee.

In this way, trusts or other bodes could be created that would use compensatory
funds to create sources of food, shelter, or medical treatment for non-human
animals that are exposed to risk.

\subsubsection{Measuring Amount}

Measuring the amount of compensation owed is a thorny problem for humans and
non-humans alike.  Compensating harm or even life seems like it requires
putting a price on dignity, anathema to core Kantian commitments.  Courts today
still use clunky, morally jarring tools like future earnings totals to
determine compensation for a wrongful death.

This is a difficult question that deserves a much longer treatment of its own.
I will acknowledge the concern, but not seek to establish a view on it here.
