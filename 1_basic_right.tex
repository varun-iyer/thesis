\chapter{Basic Rights}

	The aim of this work depends on the idea that we owe at least some moral
	duties to nonhuman animals. In this section, I will substantiate this view.
	The most important philosophical choice I make in this section is to adopt
	a rights-based approach to animal ethics as opposed to a consequential
	approach or a capabilities-oriented approach.

	To explain this choice, I’ll describe the competing views, explain the
	appeal of a rights-based approach, and provide a brief account for a
	rights-based view of nonhuman animal ethics.

	\section{Capacities-Oriented Approaches}	

	In \emph{Animal Ethics in Context}, Clare Palmer identifies three common
	capacities approaches to animal ethics: consequential approaches,
	rights-oriented approaches, and capabilities approaches.\autocite[Ch.
	2]{palmer_animal_context}

	Plamer refers to these as \emph{capacities-oriented approaches} because
	they focus on what we owe to animals by virtue of their existence and
	fundamental capacities alone. There is a distinction between
	capacities-oriented approaches and relational approaches, which describe
	what we owe to animals by virtue of the relationships we have to them.
	Both approaches are important and reflect considerations in different
	dimensions of animal ethics.

	\subsection{Consequential Views}

	Modern consequential approaches to animal ethics can trace their roots to
	Peter Singer’s influential \emph{Animal Liberation}. Singer’s utilitarian
	approach takes suffering to be the threshold criterion for moral
	consideration.  The pain of non-human animals should have equal moral
	significance to the equal pain of humans. Singer’s approach is also known
	for the \emph{equal consideration of unequal interests.} All interests
	count equally, whether human or non-human, but different creatures have
	different interests. For example, Singer claims that, humans have greater
	interests in continued life because they have long-term life projects
	whereas many non-human animals do not.

	This approach is compelling and historically significant. There is also great
	appeal in setting suffering as the threshold characteristic for a creature 
	to have moral patienthood, and a pain threshold is consistent with
	the rest of this work.

	However, Singer’s consequentialist approach is vulnerable to the same sharp
	criticisms as a consequential approach towards human ethics. Martha
	Nussbaum identifies three key theoretical problems: (1) aggregation and
	trade-off against different kinds of goods, (2) aggregation and trade-off
	over different beings, and (3) excessive focus on satisfaction at the
	expense of agency.\autocite[71]{nussbaum_frontiers} Christine Korsgaard
	highlights highly counterintuitive outcomes of consequentialism with
	replaceability.\autocite[11.9.2, 12.3.4]{korsgaard_fellow_creatures}
	 
	Palmer identifies some bullets that a consequentialist view must bite.
	Most significant among them is the unending obligations to harness,
	control, and reduce harm to all wildlife. Consequentialism runs into
	similar problems as applied to problems of self-defense. An avowedly
	consequential view excludes any discussion of liability or self-defense,
	leading to all kinds of unintuitive outcomes in defense cases. Admittedly,
	consequential approaches have a much easier time handling questions of risk
	than non-consequential views. However, there are serious moral problems
	with aggregating and distributing risks so long as the benefits are
	sufficient (which is the modern policy mantra). These will be discussed in
	greater detail later on.

	\subsection{Capabilities-oriented Views}

	The capabilities approach to animal rights was first articulated in Martha
	Nussbaum’s work.\autocite{nussbaum_frontiers} The capabilities approach has
	moral foundations in wonder, and the recognition of dignity in other forms
	of life.  The approach demands that non-humans must have the opportunity to
	exercise their fundamental capabilities which are necessary for them to
	lead flourishing lives.

	The capability approach has some appealing upshots that resolve concerns
	about utilitarianism.  The approach does not endorse trade-offs or
	comparisons between different \emph{kinds} of good, and the approach finds
	value in the creature and life itself and not merely in experience, as the
	utilitarian does.

	However, Nussbaum’s application of the capabilities approach is not always
	consistent,\autocite[404]{nussbaum_frontiers} and the approach has more to
	say about global justice than about how to conduct interpersonal
	interactions.  One weakness of the capabilities view is that it does not
	identify a difference between action and inaction with respect to
	satisfying fundamental capabilities. Like the utilitarian view, this leaves
	us with endless obligations to control and support wildlife. Some embrace
	this consequence, but I see it as a \emph{reductio} of the view.  The
	capabilities approach is not the one that I adopt or endorse here, but I
	think that it is largely consistent with the rest of this text, and could
	be used as a starting point for this project.

	\subsection{Animal Rights}

	The rights-based approach to animal ethics is most closely identified with
	Tom Regan and Gary Francione\footnote{\emph{See e.g.}
	\cite{palmer_animal_context}}. More recently, Kant scholars Christine
	Korsgaard and Barbara Herman have also articulated theories of animal
	rights.\footnote{\cite{korsgaard_fellow_creatures} TODO herman}

	Rights approaches are distinct from other approaches because they focus on
	identifying inviolable (or strongly protected) guardrails on moral
	activity.  Rights theorists identify impermissible actions (e.g. killing an
	animal) but distinguish between acting (and thus violating a guardrail or a
	rule) and merely allowing to occur (perhaps failing duties of aid, but not
	of equal moral weight to acting).  Rights theorists can also best explain
	liability, agent-relative prerogatives, and other features of self-defense.
	It is a little less certain how rights theories may apply in conditions of
	uncertainty, but rights views can immediately identify the intuitive
	discomfort with treating risks to life as mere costs to optimize.

	Any rights-based view could be compatible with this work. In order to
	motivate the view before diving into novel work, I will briefly give
	Christine Korsgaard’s Kantian, rights-based approach of animal rights.

	\section{Korsgaard’s Kantian View of Animal Rights}
		
	For the purposes of this work, I take the view of fundamental rights
	advanced by Korsgaard in \emph{Fellow
	Creatures}.\autocite{korsgaard_fellow_creatures} Much of the work presented
	later could be adapted to suit many non-consequentialist theories, but I
	will present Korsgaard’s view to motivate a basic rights approach to animal
	ethics.

	Korsgaard’s argument is established on the foundation of Kant’s moral
	philosophy.  Kant seeks to determine the presuppositions of valuing; the
	premises that are implicitly accepted any time we confer value upon
	something.  There is no metaphysical “absolute” reference frame from which
	“true” value can be determined.  Rather, things are valuable insofar as
	they are valuable to a being.

	Kant argues that rational beings can only pursue ends that are absolutely
	good, good from all perspectives.\autocite[8.4.1]{korsgaard_fellow_creatures} So, an end that one
	rational being pursues is absolutely good and worthy of pursuit by any
	other.  Humans have the capability to act as rational beings.  When humans
	rationally act on ends, they confer absoulute value on those ends, marking
	them as good absolutely.  “By pursuing what is good for you as good
	absolutely, you show that you regard yourself as an end in itself, or
	perhaps to put it a better way, you make a claim to that
	standing.”\autocite[8.4.4]{korsgaard_fellow_creatures}

	This seems to leave animals --- most of whom are not rational beings ---
	out of the moral picture.  If animals are not rational, then they cannot
	make a claim to standing as an end in themselves.

	Korsgaard argues that Kant uses the phrase ‘end in itself’ to refer to two
	slightly different concepts.  Kant somtimes refer to an ‘end in itself’ as
	a being who has the ability to legislate for itself and all rational
	beings.  At other times, Kant refers to an ‘end in itself’ as a being whose
	ends are recognized as absolutely good and protected by universal
	legislation.  Korsgaard argues that the two do not always need to be one
	and the same.

	When we act as rational beings, we do not assert that only rational beings
	have value.  After all, a rational being without any substance, form, or
	other natures would not have any ends within itself to seek or any desires
	to pursue.  Instead, our rational nature confers value upon the ends we
	seek as animals, creatures who have representations of the world and seek
	good within it.  We share these ends and this nature with other non-human
	animals, and so when we value ourselves as creatures that have a final
	good, we also confer value on other creatures with final goods.  In her own
	words,

	\begin{quote} As rational beings, we need to justify our actions, to think
		there are reasons for them. That requires us to suppose that some ends
		are worth pursuing, are absolutely good. Without metaphysical insight
		into a realm of intrinsic values, all we have to go on is that some
		things are certainly good-for or bad-for us. That then is the starting
		point from which we build up our system of values—we take those things
		to be good or bad absolutely—and in doing that we are taking ourselves
		to be ends in ourselves.  But we are not the only beings for whom
		things can be good or bad; the other animals are no different from us
		in that respect. So we are committed to regarding all animals as ends
		in themselves.\autocite[8.5.5]{korsgaard_fellow_creatures} \end{quote}

	\subsection{Korsgaard \& Using Animals}

		Thus far, it has been established that animals are ends in themselves.
		But it has not yet been explained what duties are generated by this
		claim, or what it means to treat an animal as end in itself.

		Specifically, there is some ambiguity about whether Korsgaard finds it
		especially wrong to \emph{use} an animal as opposed to merely harming
		them.  The use/harm distinction is important to this work because moral
		intuitions and academic writing regarding self defense generally relies
		on distinctions between use, intentional harm, and merely foreseen but
		unintended harm.  Most famously, it is easier to argue that it is
		permissible to pull the lever on the trolley problem (foreseen but
		unintended harm), and more difficult to argue that it is permissible to
		push a person’s body in front of the trolley (use). Different authors
		identify different morally relevant features that distinguish these
		cases, but in general, most non-consequential theories of defense rely
		on some distinctions like these.

		I believe that there is a great moral difference between killing
		gophers that eat humans’ vegetables and using rats for medical
		experimentation, even if they both undergo virtually the same
		treatment. Making this case requires a use/harm distinction for the
		treatment of nonhuman animals.

		It is not absolutely clear to me whether Korsgaard herself believes
		that there is a use/harm distinction for non-human animals. I believe
		that there is such a distinction, but I will briefly describe some of
		the contradictory indications in Korsgaard’s work.

		It is clear that Korsgaard sees some differences between treating
		humans as ends and animals as ends because humans are co-equal
		legislators while animals are merely protected by legislation.
		Korsgaard comes to the conclusion that animals have a good, but because
		they lack rationality, they do not get an ‘equal vote’ in our
		interactions with them.\autocite[12.2.1]{korsgaard_fellow_creatures}
		Despite the fact that they cannot consent to our interactions with
		them, because our own moral legislation declares their ends to be
		worthy, we have a duty to treat animals in a manner that is consistent
		with their ends.\autocite[12.2.1]{korsgaard_fellow_creatures} Korsgaard
		distinguishes this treatment from treatment of animals “in ways to
		which they would consent if they could,” but she is unclear if the two
		generate different
		outcomes.\autocite[12.2.1]{korsgaard_fellow_creatures}

		Korsgaard uses the difference between animals and human ends to
		differentiate herself from Reagan’s strong position against using
		animals:
		\begin{quote}
			Recall Tom Regan’s remark, quoted in 10.2.1, that “What’s
			wrong— fundamentally wrong—with the way animals are treated isn’t
			the details that vary from case to case. It’s the whole system. The
			fundamental wrong is the system that allows us to view animals as
			our resources, here for us.” Since I think we are treating animals
			as ends in themselves in the sense that they are ends in themselves
			if we treat them in ways that are compatible with their good, I do
			think we need to think about “the details that vary from case to
			case.”\autocite[12.2.1]{korsgaard_fellow_creatures}
		\end{quote}
		This suggests to me that Korsgaard does \emph{not} find that the use of
		animals is in itself wrong; using an animal as a resource in a way that
		is consistent with its good could be permissible.

		Later on, however, Korsgaard makes a comparison between animal testing
		and self defense: “If a madman has got hold of a gun and is threatening
		to shoot your child, many of us think you may shoot the madman if it is
		the only way to save your child. But you may not steal the madman’s
		organs even if your child will die without a
		transplant.”\autocite[12.5.2]{korsgaard_fellow_creatures}

		This suggests to me that Korsgaard would believe at least that
		\emph{use} of a creature is more difficult to justify than merely
		harming them. These views are in tension, but need not be
		contradictory.  Korsgaard might reconcile them by arguing that
		\emph{pure} use of a creature (with no attendant losses to its end) is
		not wrong at all, but that using a creature in a way that causes harm
		is more difficult to justify than the harm alone.

		Another approach might be to argue that use of a creature in a way that
		does not detract from any of their ends is not really use, in the same
		way that taking advantage of an employee’s labor is not using them so
		long as they consent. The parallel between consent for humans and the
		good of animals also lines up nicely with Korsgaard’s earlier thoughts
		on the difference between treating animals in ways that are good for
		them and treating animals in ways to which they can consent.
		If the former and latter are coextensive, we can replace consent in
		human cases with ‘preserving good’ in animal cases, thus explaining why
		cases of ‘pure’ use are not really use at all in the animal case.

		Both of these routes are defensible, consistent with Korsgaard, and
		agreeable to the rest of my work, so I will not pursue an exhaustive
		argument here.

	\section{Implications and Uncharted Territory}
		The ‘basic’ starting point that I take here and its implications are
		not uncontroversial, even among some animal ethics scholars.

		The least of the implications that Korsgaard points out are the
		abolition of all (or nearly all) animal husbandry and farming, or at
		least economically viable farming practices. It would also demand the
		abolition of all medical experimentation on animals, although some
		behavioral studies that did not treat animals contrary to their
		well-being in any way may still be permissible.

		Korsgaard also advocates for a preservationist ethic towards wild
		animals. While the precise duties and changes in behavior this might
		generate are yet unexplored, it is clear that the relentless pace of
		human expansion depends on the moral assumption that wild animals are
		not ends and do not deserve any consideration. For example, U.S.
		Environmental Impact Statements do not require examination of potential
		harm to ‘pest’ or common wild animals like gophers or
		deer.\footnote{cite}

		While the question of expansion and its impact on wildlife is certainly
		interesting, it has already been explored to some degree from a few
		different perspectives. Donaldson \& Kymlicka identify wild animals as
		sovereign beings that have rights to their own space, which Katherine
		Bradshaw supports with a legal theory of wild animals as property
		owners. Wayne Gabbardi outlines a political ethic that eschews modern
		capitalist expansionism and advocates for animal voices to have a seat
		at the table of community decision making. All of these views build
		forward from the idea that animals are ends that have certain rights
		that deserve moral consideration.

		In comparison, what has been explored less is our duties to
		\emph{liminal} animals; creatures like deer, mice, racoons, birds,
		and gophers.  Animals of these species often live in or interact with
		human spaces.  Our interests only conflict with domestic and wild
		animals when we seek to \emph{use} their bodies or their spaces for our
		purposes. In contrast, our interests are constantly on a collision
		course with the interests of liminal animals. They nest in homes, eat
		gardens, and tunnel through grass. In return, they are trapped,
		starved, run over by automobiles, or collide with glass buildings.

		These animals too are ends in themselves, but it is less clear what our
		duties are towards them. It seems clear to me (and
		Korsgaard)\footnote{cite} that humans do not have a duty to end their
		own lives or seek to reduce our species to nothing. However, this
		radical approach would be necessary to truly eliminate all of the harm
		that we cause to these creatures.

		While Donaldson \& Kymlicka, Wayne Gabardi, and Clare Palmer all
		address this problem and agree that the harms that we impose should be
		reduced, they do not explore the problem thoroughly.  Specifically,
		they fail to grapple with the potentially unending duties that might be
		generated by taking the idea that we ought not harm animals to the
		fullest extent.

		In this work, I seek to sketch out some of our duties to liminal
		animals, and conversely, some things that are not within our duties to
		these creatures.
