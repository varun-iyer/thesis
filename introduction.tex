\chapter*{Introduction}
 \addcontentsline{toc}{chapter}{Introduction}

Animal ethics has an overwhelming focus on animals used for food and animals
used in research, and rightfully so. The majority of the animals that humans
ever interact with are livestock animals --- everything else is within the
statistician’s margin of error.\footnote{sullivan and wolfson} For nearly all
applied purposes, animal ethics is farmed animal ethics.

The majority of literature in animal ethics (especially literature by
philosophers who do not focus on animal ethics) has been focused on
capacity-oriented views towards animals. These kinds of views focus on what our
duties are to non-humans merely because they can suffer/experience/have ends.
These views are the basic first steps necessary to do any other animal ethics,
and in general, these views are sufficient to denounce modern animal farming.

There is also a considerable amount of discussion about our duties to wild
animals. Environmental ethicists discuss duties to wild animals from the
perspective biotic community integrity, and animal ethicists debate whether our
approach to wild animals should be interventionist or preservationist.

But the wild animal/domestic animal discussions leave behind an excluded middle
(pun intended). Animals like deers, rats, squirrels, pigeons, seagulls, and
gophers share human-modified spaces but are not under direct human control. We
interact with these animals often, and our duties towards them cannot be
described in terms of ‘biotic communities’ or ‘preservationism’ since our
day-to-day activities affect their lives greatly. These animals are alternately
referred to as liminal animals or contact-zone animals. I will
treat these terms as interchangeable for the purpose of this paper.

Just three major works discuss contact zone animals: Clare Palmer’s
\emph{Animal Ethics in Context}, Sue Donaldson and Will Kymlicka’s
\emph{Zoopolis}, and Wayne Gabardi’s \emph{The Next Social Contract}.  While
all agree that our treatment of liminal animals should improve, none of them
enumerate or discuss specific ways that contact-zone animals are harmed by
humans or what our duties are to avoid this harm.

In this paper, I will examine two human interactions that are harmful to
liminal animals: defense of human property (i.e. trapping gophers) and imposing
risks (i.e. driving cars). I will establish the duties that we have to moderate
these interactions.

In Part 1, I will outline a view of basic, capacity-oriented duties towards
animals that I will use as a starting point. In Part 2, I will discuss a theory
of property and the circumstances in which humans can defend property rights
against nonhuman animals. In Part 3, I will discuss the risks of harm that
humans impose on non-human animals and the limits of those risks.
